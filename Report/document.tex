%%This is a very basic article template.
%%There is just one section and two subsections.
\documentclass{article}

\usepackage{amsmath}
\usepackage{algorithm}
\usepackage{algpseudocode}

\floatname{algorithm}{Procedure}
\renewcommand{\algorithmicrequire}{\textbf{Input:}}
\renewcommand{\algorithmicensure}{\textbf{Output:}}


\title{Min-max Approach in Decoding}
\begin{document}

\maketitle
\section*{Introduction}

Most of the problems in NLP try to solve following problem, 
\begin{align*}
 y^* = \arg\max_{y\in\mathcal{Y}} F(y)
\end{align*}
where $\mathcal{Y}$ is the set of all possible structures, and $F(y)$ is the
score function. Note that if score function can be written as sum of functions
over subsets of $y$, then problem becomes
\begin{align*}
y^* = \arg\max_{y\in\mathcal{Y}} \sum_i f_i(y_i)
\end{align*}
When problem have that structure, we will discuss how to solve a different
problam which is,
\begin{align*}
y^* = \arg\max_{y\in\mathcal{Y}} \min_i f_i(y_i)
\end{align*}

\section*{POS and Viterbi}


\begin{algorithm}
\caption{Viterbi for max-min problem}
\begin{algorithmic}
	\State{\textbf{Input:} A sentence $x_1,\dots,x_n$ parameters $q(s|u,v)$ and
$e(x|s)$}
	\State \textbf{Definitions:} Define $\mathcal{K}$ to be the set of possible tags.
Define $\mathcal{K}_{-1} = \mathcal{K}_{0} = \{*\}$, and $\mathcal{K}_k =
\mathcal{K}$ for $k=1\dots n$.
	\For{$k \gets 1 \textrm{ to } n$}
		\For {$u \in \mathcal{K}_{k-1} \textrm{ and } v \in \mathcal{K}_k$}
			\State \[\pi(k,u,v) = \max_{w \in \mathcal{K}_{k-2}} \min \{\pi(k-1,w,u),
			q(v|w,u)+e(x_k|v)
			\}
			\]
		\EndFor
   \EndFor
   \State \textbf{return} $\max_{u \in \mathcal{K}_{k-1}, v \in
   \mathcal{K}_k} \min\{\pi(n,u,v),q(\text{STOP}|u,v)\}$
\end{algorithmic}
\end{algorithm}

Even though changes to original Viterbi algorithm are minimal, the above
algorithm brings out an important uncertanity based on the fact that our problem
does not have a unique solution. There can be many different solutions to
max-min objective and the above algorithm would pick one randomly. On the other
hand if we use the hypergraph representation of the dynamic program, we can
prune that graph so that resulting graph would contain only the edges that are
in an optimal solution. After constructing the hyper-graph, achieving that goal
is actually quiet simple. 
\begin{itemize}
  \item Find \[w^* = \max_{y\in\mathcal{Y}} \min_i f_i(y_i)\]
  \item Delete all edges $e$ s.t. $weight(e)< w^* $
\end{itemize}
Any path from root to end in the remaining graph will be a solution to $y^* =
\arg\max_{y\in\mathcal{Y}} \min_i f_i(y_i)$. This is actually quiet easy to
observe. Assume there is a path that is not solution to our objective, that
would mean it contains an edge $e$ s.t. $weight(e)< w^* $, however, we already
prunned those edges, so there cannot be such a path. Also note that there exists
at least one path from root to end in the remaining graph, since we found $w^*$
using our modified viterbi, and we did not prune any of the edges cotained in
that path.\\

After that point we can try different ideas to find a solution. A simple idea
would be to run generic viterbi algorithm (or shortest path) on remaining
hypergraph and get a solution. Another idea would be setting weight of all the
edges $weight(e)=w^*$ to infinity, and running our algorithm would give the
solutions that maximize lowest, and second lowest weighted edges in the path. We
can even keep doing that until we get a single solution.
\end{document}
